\documentclass[a4paper,12pt, oneside]{book}
\usepackage[czech]{babel}
\usepackage[utf8]{inputenc}

% Plugin na sloupce
\usepackage{multicol}

% Plugin na obrazky
\usepackage{graphicx}

% Toto je plugin diky kteremu pujde klikat do obsahu
\usepackage{hyperref}
\hypersetup{
    colorlinks,
    citecolor=black,
    filecolor=black,
    linkcolor=black,
    urlcolor=black
}

% Plugin na hezci nadpis kapitoly
% https://texblog.org/2012/07/03/fancy-latex-chapter-styles/
\usepackage[T1]{fontenc}
\usepackage{titlesec, blindtext, color}
\definecolor{gray75}{gray}{0.75}
\newcommand{\hsp}{\hspace{20pt}}
\titleformat{\chapter}[hang]{\Huge\bfseries}{\thechapter\hsp\textcolor{gray75}{|}\hsp}{0pt}{\Huge\bfseries}


% Zakladni info o dokumento
\title{Ročníková práce}
\author{Lukáš Dulík}
\date{\today} % Tady sa pak moze nastavit jine datum, ted to da vzdycky aktualni


% Plugin na zahlavi zapati
\usepackage{fancyhdr}


\fancyhf{}
\lhead{Strojové učení}
\lfoot{Ročníková práce KVARTA}
\rfoot{\thepage}

% Dulezite pro spravne zobrazovani zahlavi zapati
\pagestyle{empty}
\fancypagestyle{plain}{}

\begin{document}

% Zrusi cislovani na zacatecnich strankach
\pagenumbering{gobble}

\begin{titlepage}
    \begin{center}
        \vspace*{1cm}

        \Huge
		Gymnázium Jana Pivečky a Střední odborná škola Slavičín \\

		% Logo GJP
		\includegraphics[width=0.4\textwidth]{img/gjp.png}

        \textbf{Ročníková práce}

        \vspace{0.5cm}
        \LARGE
        Téma: Strojové učení
    \end{center}

	\vspace{1.5cm}

	% Vyplni zbyvaji prostor
	\vfill

	\vspace{0.8cm}

	\vspace{5pt}
	% Cara nahore
	\hrule
	\vspace{6pt}

	\Large

	\makeatletter
	\begin{multicols}{2}
		\noindent
		Slavičín \\
		Datum: \@date

	\columnbreak
		\noindent
		\null\hfill Třída: Kvarta \\
		\null\hfill \@author
	\end{multicols}

	\makeatother

	\vspace{5pt}

	% Cara dole
	\hrule

\end{titlepage}

\newpage
\mbox{}
\newpage


\noindent
Prohlašuji, že jsem ročníkovou práci vypracoval samostatně
a výhradně s použitím citovaných pramenů.

Ve Slavičíně: dne datum jméno, vlastnoruční podpis


\newpage
\section*{Abstrakt}
Abstract goes here

\newpage
\section*{Poděkování}
Děkuji paní x za pomoc s prací.

% Udela obsah
\tableofcontents

\clearpage
% Zacatek cislovani stranek
\pagenumbering{arabic}

\chapter{Teoretická část}



\section{Metody a algoritmy}

\subsection{Rozdělení}

\subsubsection{Učení pod dohledem}

Programu se dává datová sada, která obsahuje jak vstupy, tak požadované výstupy. Cíl programu,
je naučit se, jak obecně zmapovat vstupy na výstupy. Když program dojde k výsledným pravidlům,
jde o něco, čemu říkáme model.

\subsubsection{Učení bez učitele}

Datová sada neobsahuje požadované výstupy, vstupy nejsou nijak označkované. Program musí sám najít nějaký
způsob, jak	strukturovat informace k sobě. Cíl je najít nějaké skryté obdobné vzory.


\subsubsection{Zpětnovazební učení}

Program pracuje v nějakém dynamickém prostředí, ve kterém musí splnit určitý cíl.
Zpraidla jde o věci jako řízení auta na cestě nebo hraní hry proti protivníkovy.
Když ten úkol zkouší řešit, dostává zpětnou vazbu, která je úměrná
tomu, jak dobře problémy zvládá. Cíl je tedy hodnotu této zpětné vazby maximalizovat.

\subsubsection{Ostatní metody}

Byli vyvinuty další metody strojového učení, které nespadají do ani jedné z těchto
tří kategorií. V systémech je často kombinovánno víc způsobů dohromady.

\subsection{Regresní analýza}

Jde o jednoduchou formu učení pod dohledem. Obecně jde o způsoby, jak dosadit nějakou
přímku nebo křivku do grafu bodů. Snažíme se odhadnout hodnotu jisté náhodné veličiny
(též cílová proměnná, regresand, či vysvětlovaná proměnná) se znalostí jiných veličin
(nezávislých proměnných, vysvětlujích proměnných anebo kovariát).

\section{Využití}






\chapter{Praktická část}

V rámci praktické části jsem se rozhodl naprogramovat dvě ukázky
základních problémů strojového učení pod dohledem: regrese a klasifikace.

Pro ukázku regrese jsem vytvořil program vizualizující lineární regresi.
V tomto programu můžete klikáním přidávat datové body. Pomocí lineární regrese
vždy vypočítá parametry přímky, která proloží všechny body z grafu. Tato přímka
je pak vykreslena čárkovaně.
Pro implementaci jsem použil jazyk Python, protože je jednoduchý a má výbornou
knihovnu matplotlib, která dělá vykreslování grafů záležitostí jednoho řádku.

Dále jsem naprogramoval klasifikační program číslic. Ve svém nitru používá
záladní formu umělé neuronové sítě.
Chtěl jsem se naučit, jak fungují neuronové sítě a tak jsem se rozhodl si
ji naprogramovat sám. Rozpoznávání číslic jsem si zvolil, protože je
to takový základní úkon v oblasti neuronových sítí. Díky tomu jsou
na internetu volné datové sady označkovaných číslic. Já jsem si vybral
volně dostupnou databázi MNIST od Národního institutu standardů a technologie USA,
která obsahuje 60 000 trénovacích obrázků a 10 000 trénovacích obrázků v rozlišení
28 na 28 pixelů. Každý obrazek má k sobě informaci, o jakou číslici se doopravdy jedná.
Přemýšlel jsem, zda to udělat v Pythonu, nebo v C++ a nakonec jsem se
rozhodl pro jazyk C++, a to kvůli rychlosti, přísnosti a uzpůsobilosti k objektově
orientovanému programování.

Myšlenka fungovaní neuronové sítě je celkem komplexní na pochopení. Pochopit to nebyla
zrovna záležitost několika minut. Učil jsem se podle videí Granta Sandersona (3Blue1Brown)
a každé video jsem si musel pustit několikrát znova, abych ten koncept nějak dostal do hlavy.
Chtěl jsem, aby to fungovalo rychle, a tak jsem se snažil každou část algoritmu zpracovávat
pomocí lineární algebry. Například násobení matic a vektorů je mnohem rychlejší než rekurzivní
operace s každým neuronem, díky tomu, že jsou na to procesory i násobící algoritmy skvěle
optimalizované. Na lineární algebru jsem si vybral knihovnu Eigen, která má typy pro
vektory a matice a podporuje všechny operace. Přišla mi jako nejjednoduší knihovna na
lineární algebru, její nevýhoda však je, že nepodporuje akceleraci výpočtů pomocí grafické
karty. S matematickou interpretací trénovacího algoritmu za pomocí operací
lineární algebry mi pomohl kamarád.








\chapter{Závěr}

Jo jo bla bla shrnutí

\chapter{Seznam použité literatury}
Bishop, C. M. (2006), Pattern Recognition and Machine Learning, Springer, ISBN 978-0-387-31073-2

\chapter{Seznam odkazů}

Grant Sanderson - Neuronové sítě:
https://youtube.com/playlist?list=PLZHQObOWTQDNU6R1_67000Dx_ZCJB-3pi



\chapter{Seznam obrázků}


\chapter{Seznam příloh}


\end{document}

